\documentclass[a4paper]{article}

\usepackage[verbose]{xsim}
\usepackage{xsim-utils}


\DeclareExerciseEnvironmentTemplate{exe}{%
    \subsection*{%
        \XSIMmixedcase{\GetExerciseName}\nobreakspace
        \GetExerciseProperty{counter}%
        \normalfont%
        \GetExercisePropertyT{subtitle}{ {\itshape\PropertyValue}}%
        \nobreakspace($\GetExercisePropertyTF{points-sum}{\PropertyValue=\printgoal{\PropertyValue}}{???}$~\XSIMtranslate {point-abbr})%
        (ID \GetExerciseProperty{id})
    }
}{}

\DeclareExerciseEnvironmentTemplate{subexe}{%
    \item[(\GetExerciseProperty{full-counter})] %
    \GetExercisePropertyT{points}{%
        (\printgoal{\PropertyValue}~\XSIMtranslate{point-abbr})%
    } %
    (ID $\GetExercisePropertyT{container}{\PropertyValue.}\GetExerciseProperty{id}$)
}{}

\DeclareExerciseType{subexercise}{
    exercise-env = subexercise,
    solution-env = subsolution,
    exercise-name = Unterfrage,
    solution-name = Unterantwort,
    exercise-template = subexe,
    solution-template = subexe
}

\DeclareExerciseType{subsubexercise}{
    exercise-env = subsubexercise,
    solution-env = subsubsolution,
    exercise-name = Unterunterfrage,
    solution-name = Unterunterantwort,
    exercise-template = subexe,
    solution-template = subexe
}

\xsimsetup{
    exercise/template = exe,
    solution/template = exe,
    solution/print = true,
    %
    subsolution/print = true,
    subexercise/within = exercise,
    subexercise/the-counter = \alph{subexercise},
    %
    subsubsolution/print = true,
    subsubexercise/within = subexercise,
    subsubexercise/the-counter = \roman{subsubexercise}
}

\XSIMloadmodules{nested,nested-points}

\XSIMNestedPointsSetup[points]{exercise}
\XSIMNestedPointsSetup{subexercise}
\XSIMNestedPointsSetup{subsubexercise}

\begin{document}

\gradingtable

\begin{exercise}[
  subtitle={Erste Aufgabe},
  points={1+2}
]
    Erste Aufgabe
    \begin{enumerate}

        \begin{subexercise}[points={3}]
            Erste Unteraufgabe
        \end{subexercise}
        \begin{subsolution}
            Erste Unterlösung
        \end{subsolution}

        \begin{subexercise}
            Zweite Unteraufgabe
        \end{subexercise}
        \begin{subsolution}
            Zweite Unterlösung
        \end{subsolution}

        \begin{subexercise}[points={5}]
            Dritte Unteraufgabe
        \end{subexercise}
        \begin{subsolution}
            Dritte Unterlösung
        \end{subsolution}

    \end{enumerate}
\end{exercise}
\begin{solution}
    Erste Lösung
\end{solution}

\begin{exercise}[
  subtitle={Zweite Aufgabe mit langem Titel}
]
    Zweite Aufgabe
    \begin{enumerate}

        \begin{subexercise}[points={8}]
            Vierte Unteraufgabe
        \end{subexercise}
        \begin{subsolution}
            Vierte Unterlösung
        \end{subsolution}

        \begin{subexercise}[points={9}]
            Fünfte Unteraufgabe

            \begin{enumerate}

                \begin{subsubexercise}[points={1}]
                    Einundfünfzigste Unteraufgabe
                \end{subsubexercise}
                \begin{subsubsolution}
                    Einundfünfzigste Unterlösung
                \end{subsubsolution}
                \begin{subsubexercise}[points={0}]
                    Zweiundfünfzigste Unteraufgabe
                \end{subsubexercise}
                \begin{subsubsolution}
                    Zweiundfünfzigste Unterlösung
                \end{subsubsolution}
                \begin{subsubexercise}[points={3}]
                    Dreiundfünfzigste Unteraufgabe
                \end{subsubexercise}
                \begin{subsubsolution}
                    Dreiundfünfzigste Unterlösung
                \end{subsubsolution}

            \end{enumerate}

        \end{subexercise}
        \begin{subsolution}
            Fünfte Unterlösung
        \end{subsolution}

        \begin{subexercise}[points={5+5}]
            Sechste Unteraufgabe
        \end{subexercise}
        \begin{subsolution}
            Sechste Unterlösung
        \end{subsolution}

    \end{enumerate}
\end{exercise}
\begin{solution}
    Zweite Lösung
\end{solution}

\end{document}